En México se han realizado importantes trabajos, cuyo objetivo es el de contar con una canasta que refleje las necesidades primordiales de los mexicanos; entre ellos destacan la Canasta Normativa de Satisfactores Esenciales (CNSE), la Canasta Alimentaria propuesta por la Comisión Económica para América Latina y el Caribe (CEPAL) y el INEGI, y las canastas Alimentaria y No alimentaria del CONEVAL.

La CNSE fue desarrollada con base en la Encuesta de Ingresos y Gastos Familiares de 1975 del Centro Nacional de Información y Estadísticas del Trabajo, dicha canasta buscaba reflejar el consumo frecuente de los mexicanos, y las normas vigentes y objetivos a alcanzar de la legislación mexicana. La Canasta Alimentaria que contruyó el CEPAL y el INEGI en 1998 buscaba reflejar las necesidades nutrimentales de la población mexicana, como parte de un esfuerzo para equiparar la metodología de la medición de la pobreza en América Latina. Por último, la Ley General de Desarrollo Social (LGDS), promulgada el 20 de enero de 2004, estableció la creación del CONEVAL con el objeto de normar y coordinar la evaluación de las políticas y programas de desarrollo social que ejecutan las dependencias públicas, y establecer los lineamientos y criterios para la definición, identificación y medición de la pobreza. En 2010, el CONEVAL publicó para los ámbitos rural y urbano, una canasta alimentaria construida mediante la adaptación de la metodología desarrollada por la CEPAL y el INEGI, así como una canasta no alimentaria basada en la propuesta metodológica de Hernández Laos. Estas canastas se construyeron con el objetivo de medir la pobreza multidimensional, una de las atribuciones del CONEVAL.